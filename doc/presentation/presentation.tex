\documentclass[professionalfonts]{beamer}
\usepackage{cmap}
\usepackage[T1]{fontenc}
\usepackage{lmodern}
\usepackage[french]{babel}
\usepackage{minted}
\usepackage{tikz}

\usetheme{AnnArbor}
\usecolortheme{spruce}

\makeatletter
\newcommand*{\currentname}{\@currentlabelname}
\makeatother

\newcommand{\keyword}[1]{{\color{green!50!black}\bf{#1}}}
\newcommand{\type}[1]{{\color{blue}#1}}
\newcommand{\node}{\keyword{node}}
\newcommand{\pre}{\keyword{pre}}
\newcommand{\where}{\keyword{where}}
\newcommand{\when}{\keyword{when}}
\newcommand{\merge}{\keyword{merge}}
\newcommand{\rif}{\keyword{if}}
\newcommand{\relse}{\keyword{else}}
\newcommand{\whennot}{\keyword{whennot}}
\newcommand{\true}{\keyword{true}}
\newcommand{\spawn}{\keyword{spawn}}
\newcommand{\false}{\keyword{false}}
\newcommand{\extern}{\keyword{extern}}
\renewcommand{\int}{\type{int}}
\newcommand{\bool}{\type{bool}}
\newcommand{\float}{\type{float}}
\newcommand{\on}{\type{on}}

\tikzset{onslide/.code args={<#1>#2}{%
  \only<#1>{\pgfkeysalso{#2}}
}}

\title{Rustre}
\subtitle{Rustre = Rust + Lustre}
\author[J. Caspar \and A. Mathieu]{Jean Caspar \and Adrien Mathieu}
\date{Lundi 22 Janvier}

\setbeamertemplate{navigation symbols}{}

\begin{document}
\begingroup
\setbeamertemplate{headline}{}
\setbeamertemplate{footline}{}
\begin{frame}[noframenumbering]
  \titlepage
\end{frame}
\endgroup

\section{Introduction}
\subsection{Objectifs}
\begin{frame}
  \frametitle{\currentname}

  Rustre est un DSL Lustre embarqué dans Rust.\pause
  \begin{itemize}
  \item il compile vers (l'AST) Rust;
    \pause
  \item le code Lustre peut appeler des fonctions Rust;
    \pause
  \item le code Rust peut utiliser des nœuds Lustre à travers une interface de streams
    \pause
  \item le code Lustre est enrichi d'une primitive permettant de paralléliser le calcul
  \end{itemize}
\end{frame}

\subsection{Gestion d'erreurs}
\begin{frame}
  \frametitle{\currentname}

  Les erreurs d'analyse syntaxique, de typage et d'ordonnancement sont intégrées aux messages
  d'erreur de Rust.\pause
  \begin{center}
    Demo! % erreur-01.rs
  \end{center}
  \pause
  Les erreurs ne sont pas accumulées, seule la première erreur rencontrée est signalée.
\end{frame}

\section{Frontend}
\subsection{La syntaxe}
\begin{frame}[fragile]
  \frametitle{\currentname}

  La syntaxe est inspirée de Lustre, Rust-ifiée.\pause
  \begin{minted}[frame=lines,escapeinside=@@]{text}
@\node@ f(x : @\int@, c : @\bool@) = (x, y)
@\where@
    compteur : @\int@ = 0 -> @\pre@ compteur + 1,
    o : @\int@ @\on@ c = compteur @\when@ c,
    y : @\int@ = @\merge@ c {
        @\true@ => o,
        @\false@ => x @\whennot@ c,
    };
  \end{minted}
  \pause Nous avons garder une certaine distance avec la syntaxe de Rust car pas toutes les
  expressions de Rust sont supportées.
\end{frame}

\subsection{Analyse lexicale et syntaxique}
\begin{frame}[fragile]
  \frametitle{\currentname}

  L'analyse lexicale et syntaxique repose sur celle du parser Rust: on lit un stream de tokens en
  entrée, et on rend un stream de tokens.\pause\\
  La génération de celle-ci utilise la macro \mintinline{rust}{quote!}, similaire au mécanisme de
  quote de Lisp.\pause
  \begin{minted}[frame=lines,escapeinside=@@]{rust}
let token_stream = quote! {
    #vis struct #iterator_name<T> {
        iterator: T,
        node: #node_name
    }
    @.{}.{}.@
};
  \end{minted}
  \pause
  N'ayant pas trouvé de générateur de grammaires pour \mintinline{rust}{syn} (l'API d'analyse
  syntaxique de Rust), nous avons du écrire le parseur ``à la main''.
\end{frame}

\subsection{Typage}
\begin{frame}[fragile]
  \frametitle{\currentname}

  Rustre possède trois types de base:
  \begin{itemize}
  \item \bool
  \item \int
  \item \float
  \end{itemize}
  et ne permet pas de définir ses propres types.\pause\par
  En plus de ces types simples, Rustre a des types d'horloge: \(\tau\) \on{} \(c\) où
  \(c\)~:~\bool.\pause\par
  Pour des raisons de simplicité, nous n'autorisons pas de types dépendants dans les signatures
  des fonctions:
  \vspace*{-1em}
  \begin{minted}[frame=lines,escapeinside=@@]{text}
@\node@ ou_defaut(c : @\bool@, x : @\int@ @\on@ c, v : @\int@) = (o)
@\where@
    o : @\int@ = merge {
        true => x,
        false => v whennot c,
    };
  \end{minted}
\end{frame}

\section{Backend}
\subsection{Génération de code}
\begin{frame}[fragile]
  \frametitle{\currentname}

  Pour générer du code, on commence par ordonnancer les équations.\pause{} \emph{Ce qui est
    actuellement buggé...}\pause\\
  Le système de type garanti qu'il n'y a aucun accès à une valeur non-initialisée.\pause\\
  En utilisant cette propriété, on demander au compilateur Rust de ne pas initialiser certaines
  variables, quand ce n'est pas nécessaire. On peut donc exploiter les \pre{} et \verb`->`.\pause
  \begin{center}
    Demo!
  \end{center}
\end{frame}

\subsection{Appel de fonctions Rust}
\begin{frame}[fragile]
  \frametitle{\currentname}

  L'expansion des macros se fait au moment de l'analyse syntaxique d'un programme Rust. On n'a
  donc pas accès à de l'information de typage produite par Rust.\pause\\
  Or, il est nécessaire de connaître les types des valeurs manipulées:
  \begin{minted}[frame=lines, escapeinside=@@]{text}
@\node@ main() = (x)
@\where@
    x : @\int@ = g(f(1) -> @\pre@ f(2));
  \end{minted}
\end{frame}
\begin{frame}[fragile]
  \frametitle{\currentname}

  Deux solutions possibles:\pause
  \begin{itemize}
  \item ajouter une déclaration pour chaque fonction
    \begin{minted}[frame=lines, escapeinside=@@]{text}
@\extern@ f(@\int@) -> @\bool@;
@\extern@ g(@\bool@) -> @\int@;

@\node@ main() = (x)
@\where@
    x : @\int@ = g(f(1) -> @\pre@ f(2));
    \end{minted}
    \pause
  \item obliger les appels externes à être nommés
    \begin{minted}[frame=lines, escapeinside=@@]{text}
@\node@ main() = (x)
@\where@
    y : @\bool@ = f(1) -> @\pre@ f(2),
    x : @\int@ = g(y);
    \end{minted}
  \end{itemize}
\end{frame}

\begin{frame}[fragile]
  \frametitle{\currentname}

  La première solution a deux inconvénients majeurs:\pause
  \begin{itemize}
  \item elle est lourde en déclarations
    \pause
  \item
    elle empêche le polymorphisme:
    \begin{minted}{rust}
fn print_newline(x: impl Display) {
    println!("{x}");
}
    \end{minted}
  \end{itemize}
  \pause La deuxième solution ne nécessite pas de déclarations, et supporte pleinement le système
  de types complexe de Rust.\pause
  \begin{center}
    Demo! % extern-01.rs
  \end{center}
\end{frame}

\subsection{Streams et itérateurs}
\begin{frame}
  \frametitle{\currentname}
  
  La sémantique de Lustre est basée sur les streams. Cela correspond assez naturellement aux
  itérateurs de Rust, la seule différence étant que ceux-ci ne doivent pas nécessairement être
  infinis.\pause\par
  Un nœud \(f : \tau_1 \times \ldots \times \tau_n \to \tau\) est donc traduit vers une fonction
  \[\hat{f} : (\tau_1\times\ldots\times\tau_n)^*\to\tau^*\]
  \pause
  Notons qu'en Lustre, on aurait plutôt traduit ça vers \[
    \hat{f'} : \tau_1^*\times\ldots\times\tau_n^*\to\tau^*
  \]
  ce qui est \emph{moins général}.
\end{frame}

\begin{frame}[fragile]
  \frametitle{\currentname}

  Pour passer de \(\tau_1^*\times\tau_2^*\) vers \((\tau_1\times\tau_2)^*\)
  \begin{center}
    \begin{tikzpicture}
      \draw[onslide=<1>red,onslide=<2->dashed] (0,0) rectangle (1,1);
      \draw[onslide=<1>red,onslide=<2->dashed] (0,1.5) rectangle (1,2.5);
      \pause
      \draw[onslide=<2>red,onslide=<3->dashed] (1,0) rectangle (2,1);
      \draw[onslide=<2>red,onslide=<3->dashed] (1,1.5) rectangle (2,2.5);
      \pause
      \draw[onslide=<3>red,onslide=<4->dashed] (2,0) rectangle (3,1);
      \draw[onslide=<3>red,onslide=<4->dashed] (2,1.5) rectangle (3,2.5);
      \pause
      \draw[onslide=<4>red,onslide=<5->dashed] (3,0) rectangle (4,1);
      \draw[onslide=<4>red,onslide=<5->dashed] (3,1.5) rectangle (4,2.5);
      \pause
      \node at (5,1.25) {$\ldots$};
    \end{tikzpicture}
  \end{center}\par
  \uncover<6->{
    Pour passer de \((\tau_1\times\tau_2)^*\) vers \(\tau_1^*\times\tau_2^*\)
    \begin{center}
    \begin{tikzpicture}
      \draw[onslide=<10>red,onslide=<11->dashed] (0,0) rectangle (1,1);
      \draw[onslide=<6>red,onslide=<7->dashed] (0,1.5) rectangle (1,2.5);
      \onslide<7->
      \draw[onslide=<11>red,onslide=<12->dashed] (1,0) rectangle (2,1);
      \draw[onslide=<7>red,onslide=<8->dashed] (1,1.5) rectangle (2,2.5);
      \onslide<8->
      \draw[onslide=<12>red,onslide=<13->dashed] (2,0) rectangle (3,1);
      \draw[onslide=<8>red,onslide=<9->dashed] (2,1.5) rectangle (3,2.5);
      \onslide<9->
      \draw[onslide=<13>red,onslide=<14->dashed] (3,0) rectangle (4,1);
      \draw[onslide=<9>red,onslide=<10->dashed] (3,1.5) rectangle (4,2.5);
      \onslide<14->
      \node at (5,1.25) {$\ldots$};
    \end{tikzpicture}
  \end{center}}
\end{frame}

\subsection{Spawn}
\begin{frame}[fragile]
  \frametitle{\currentname}

  On ajoute, par rapport aux langages synchrones usuels, une primitive \spawn{}.\pause\\
  La sémantique de \spawn{} \(f(x_1,\ldots,x_n)\) est celle de \pre{} \(f(x_1,\ldots,x_n)\),
  sauf que le calcul se fait dans un nouveau thread.\pause
  \begin{minted}[frame=lines,escapeinside=@@]{text}
@\node@ main() = (o)
@\where@
    compteur : @\int@ = 0 ->
        @\rif@ @\pre@ c == 5 { 0 } @\relse@ { @\pre@ c + 1 },
    c : @\bool@ = compteur == 0,
    couteux : @\int@ @\on@ c = (0 -> @\spawn@ f()) @\when@ c,
    o : @\int@ = @\merge@ c {
        @\true@ => couteux,
        @\false@ => 0 @\whennot@ c,
    };
  \end{minted}
\end{frame}

\subsection{Bugs}
\begin{frame}[fragile]
  \frametitle{\currentname}

  On a quelques bugs non résolus:\pause
  \begin{itemize}
  \item le scheduler fait des choses étranges...
    \pause
  \item les horloges sont mal propagées:\pause{}
    \begin{minted}[frame=lines, escapeinside=@@]{text}
@\node@ main(c : bool) = (o)
@\where@
    o : int = merge c {
        true => (1 -> 2) when c,
        false => 0 whennot c,
    };
    \end{minted}
    sur [\false, \true, \true], renvoie [0, 2, 2] au lieu de [0, 1, 2].
  \end{itemize}
\end{frame}

\end{document}
