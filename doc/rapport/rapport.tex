\documentclass{scrartcl}
\usepackage{cmap}
\usepackage[T1]{fontenc}
\usepackage{lmodern}
\usepackage[french]{babel}
\usepackage{microtype}
\usepackage{syntax}
\usepackage{amsmath}
\usepackage{amsfonts}
\usepackage{minted}
\usepackage{subcaption}

\title{Rustre}
\subtitle{Rustre = Rust + Lustre}
\author{Jean \textsc{Caspar} \and Adrien \textsc{Mathieu}}
\date{January 29, 2023}

\DeclareMathOperator{\mspawn}{spawn}
\DeclareMathOperator{\mpre}{pre}
\DeclareMathOperator{\tint}{int}
\DeclareMathOperator{\tfloat}{float}
\DeclareMathOperator{\tbool}{bool}

\newcommand{\keyword}[1]{{\textcolor{green!50!black}{\textbf{#1}}}}
\newcommand{\type}[1]{\textcolor{blue}{#1}}
\newcommand{\node}{\keyword{node}}
\newcommand{\pre}{\keyword{pre}}
\newcommand{\where}{\keyword{where}}
\newcommand{\when}{\keyword{when}}
\newcommand{\merge}{\keyword{merge}}
\newcommand{\rif}{\keyword{if}}
\newcommand{\relse}{\keyword{else}}
\newcommand{\whennot}{\keyword{whennot}}
\newcommand{\true}{\keyword{true}}
\newcommand{\spawn}{\keyword{spawn}}
\newcommand{\false}{\keyword{false}}
\newcommand{\extern}{\keyword{extern}}
\renewcommand{\int}{\type{int}}
\newcommand{\bool}{\type{bool}}
\newcommand{\float}{\type{float}}
\newcommand{\on}{\type{on}}
\newcommand{\as}{\keyword{as}}

\newlength{\figwidth}
\newsavebox{\rulebox}
\newcommand{\typerule}[2]{%
  \savebox{\rulebox}{$\displaystyle\frac{#1}{#2}$}
  \settowidth{\figwidth}{\usebox{\rulebox}}
  \begin{subcaption}{\the\figwidth}
    \usebox{\rulebox}
  \end{subcaption}
}

\begin{document}
\maketitle{}
\tableofcontents
\pagebreak

\section*{Introduction}
Rustre est un DSL de Lustre dans Rust. Il permet d'écrire du code Lustre dans un programme Rust,
en laissant la possibilité de faire des appels transparents depuis Lustre vers Rust et
vice-versa.

Des exemples sont disponibles dans le dossier \texttt{examples}, et peuvent être executés avec
\texttt{cargo run -{}-example \textit{exemple}}.

\section{Analyse syntaxique}
L'analyse syntaxique de Rustre s'appuie sur le flux de lexèmes fournit par Rust. Ce flux ne
présage pas des lexèmes utilisés par Rustre, c'est-à-dire qu'il ne force pas Rustre à utiliser
les mêmes lexèmes que Rust. En revanche, Rust fournit directement des informations de
positionnement des lexèmes, et ces mêmes informations sont réutilisées par Rustre pour fournir à
son tour un flux de lexèmes sortant où les lexèmes sont correctement annotés, de sorte que les
messages d'erreur de Rust peuvent remonter à l'origine du problème dans le code source du
programmeur (et pas dans le code généré).

En annexe est fournit la syntaxe de Rustre. La syntaxe fournit correspond à une version lisible,
contrairement à celle dans le code source qui a été transformée de sorte à être LL(1).

\section{Typage}
Rustre utilise trois systèmes de type, comme Lustre, pour garantir la correction du code.
Cependant, Rustre n'accepte qu'un sous-ensemble très simple des programmes Lustre du point de vue
du typage, car il ne supporte pas:
\begin{itemize}
\item les types dépendants, pour le système de types d'horloge;
\item le sous-typage;
\item le polymorphisme.
\end{itemize}
\subsection{Types de base}
\subsubsection{Types de Rustre}
Rustre fournit trois types de base, à savoir
\begin{itemize}
\item \(\tint\)
\item \(\tfloat\)
\item \(\tbool\)
\end{itemize}
et permet de convertir un entier en un flottant avec \(e\ \as\ \float\) si \(e:\int\).

\subsubsection{Typage}
Le typage s'effectue en vérifiant que le type d'une variable correspond bien au type de
l'expression qu'on essaye de lui assigner. Comme tous les littéraux ont un unique type possible,
et que le type de toutes les variables (donc les types de retour des n\oe{}uds) sont
explicitement déclarées, ce n'est qu'une simple vérification: aucun mécanisme d'inférence de type
n'est nécessaire.

La seule exception est pour l'appel à des fonctions de Rust. La compilation de Rustre se faisant
pendant la phase d'analyse syntaxique du code Rust, il est impossible d'obtenir des informations
de typage sur ces fonctions. Or, il arrive que ces informations soient nécessaires, comme dans
le cas suivant
\begin{minted}[frame=lines,escapeinside=@@]{text}
@\node@ main() = (x)
@\where@
    x : @\int@ = g(f(1) -> @\pre@ f(2));
\end{minted}
En effet, Rustre doit générer une déclaration d'un type contenant la mémoire utilisée par le
\pre{}. Rustre impose donc qu'il existe une annotation de type explicite pour la valeur de retour
d'un appel de fonction externe. Comme les seules annotations de type sont au niveau des
déclarations de variables, il est nécessaire de nommer ces expressions intermédiaires:
\begin{minted}[frame=lines,escapeinside=@@]{text}
@\node@ main() = (x)
@\where@
    y : @\bool@ = f(1) -> @\pre@ f(2),
    x : @\int@ = g(y);
\end{minted}
À noter que ce n'est pas la valeur de retour de chaque appel à une fonction qui est nommé, ce qui
est permit par de l'inférence de type très basique.

Par ailleurs, Rustre ne permet pas de polymorphisme pour les types de base.

\subsubsection{Polymorphisme de Rust}
Rustre ne vérifie pas la cohérence des types des fonctions externes:
\begin{minted}[frame=lines,escapeinside=@@]{text}
@\node@ main() = ()
@\where@
    x : @\bool@ = f(true),
    y : @\int@ = f(3);
\end{minted}
En effet, il est possible que ces expressions soient bien typées, par exemple si \(f\) est
définie comme suit
\begin{minted}[frame=lines]{rust}
fn f<T>(x: T) -> T {
    x
}
\end{minted}
ce qui permet effectivement d'utiliser le polymorphisme de Rust en Rustre.

\subsection{Horloges}
Le système de type d'horloge est définie comme suit
\begin{align*}
  \kappa ::=&\; \textrm{base}\\
  |&\; \kappa\ \textrm{on}\; v & \textrm{où }v : \textrm{\bool}^\kappa\\
  |&\; \kappa\ \textrm{on}\; !v & \textrm{où }v : \textrm{\bool}^\kappa
\end{align*}
où l'on note \(e : \tau^\kappa\) si \(e\) a le type de base \(\tau\) et le type d'horloge
\(\kappa\).

L'égalité entre types d'horloge est résolue de façon purement syntaxique, ie. aucun effort n'est
fait pour déterminer si deux expressions booléennes sont sémantiquement équivalentes.

Par ailleurs, les types dépendants dans les signatures de n\oe{}uds sont interdits par Rustre, ce
qui empêche de factoriser du code. Par exemple, ce n\oe{}ud, qui pourrait être utile, ne peut pas
être définit:
\begin{minted}[frame=lines,escapeinside=@@]{text}
@\node@ ou_alors(c : @\bool@, x : @\int@ @\on@ c, v : @\int@) = (o)
@\where@
    o : @\int@ = @\merge@ c {
        @\true@ => x,
        @\false@ => v @\whennot@ c,
    };
\end{minted}

\subsection{Type d'accès}
Pour vérifier qu'aucun accès à une valeur non définie n'est effectué, chaque expression a un type
d'accès \(n\in\mathbb{N}\), qui signifie que l'expression est définie à partir de la \(n\)-ème
itération. Rustre impose que toutes les variables puissent être typées avec le type d'accès
\(0\), c'est-à-dire que les variables soient toujours bien définies.

Cette règle est plus restrictive que de demander que les variables de retour d'un n\oe{}ud soient
toujours définies, mais la vérification se fait plus facilement.

\section{Ordonnancement}
Pour l'ordonnancement, on génère une équation pour chaque sous-expression.
De plus, on génère le graphe de dépendances entre les équations.

Il y a plusieurs types d'équations: les équations qui calculent une variable, les appels de
fonctions externes et les appels de n\oe{}uds, avec ou sans \(\mspawn\). De plus, il y a
également un type d'équation spécial pour stocker l'entrée du n\oe{}ud, ainsi qu'un autre type
spécial pour effectuer stocker les résultats des expressions avec un \(\mpre\) dans la
mémoire. On parcourt ensuite le graphe des dépendances pour ordonner les équations, et s'il y a
un cycle on lève une erreur. On parcourt également le graphe des appels entre n\oe{}ud pour
interdire les appels récursifs de n\oe{}uds.

\section{Génération de code}

\subsection{Non initialisation avec les \pre}
Pour la génération de code, on génère une variable dans la fonction \texttt{step} pour chaque
équation, et on ajoute un membre à la structure représentant un n\oe{}ud pour chaque élément qui
doit rester en mémoire d'un appel à l'autre: les résultats des expressions avec \(\mpre\), et
les n\oe{}uds instanciés, avec ou sans \(\mspawn\).

Cette variable a pour type \texttt{MaybeUninit}, qui représente une zone mémoire qui n'est
potentiellement pas initialisée. Il est interdit de lire une telle zone mémoire si l'on est pas
sûr que la mémoire est initialisée, mais deux mécanismes permettent de s'assurer que c'est le
cas. D'une part, chaque équation possède une liste d'horloge, ajoutée au typage, qui spécifie
quand cette équation doit s'exécuter. De plus, elle possède une liste d'indices et de booléens
qui correspond à des équations de la forme \texttt{x = a -> b}. Pour une telle équation, on
génère une variable booléenne (que l'on garde en mémoire) qui vaut \true{} si et seulement
si c'est la première fois que l'équation s'exécute.  Une équation qui correspond à une
sous-expression de \texttt{a} référencera l'indice de \texttt{x} avec le booléen \true,
tandis qu'une sous-expression de \texttt{c} aura le booléen \false.

Ainsi, lorsqu'on exécute une équation, on vérifie que si elle apparaît comme sous-équation d'un
ou plusieurs \texttt{->}, elle fait partie d'une branche qui doit s'exécuter, et que toutes ses
horloges valent \true. Dans le cas contraire, la variable correspondante n'est pas
initialisée. Le typage de Rustre nous permet d'être sûr de ne pas lire une variable non
initialisée.

Une subtilité est que l'invariant interne dans la génération de code est qu'une variable qui
possède la valeur \texttt{MaybeUninit::uninit} (c'est-à-dire, qui n'est pas initialisée) n'est
jamais utilisée pour faire des calculs. Cependant, une variable non initialisée peut être copiée
dans une autre variable non utilisée. Or, \texttt{a = b} est un comportement non défini en Rust
si \(a\) est non initialisée (car Rust voudrait alors appeler le destructeur de la valeur en
\(a\), ce qui pourrait libérer de la mémoire en lisant le contenu de \(a\), qui est non
initialisée). Pour cela, on utilise des primitives plus bas niveau \texttt{std::ptr::read} et
\texttt{std::ptr::write}, qui sont \texttt{unsafe}.

\subsection{Sémantique de \(\mspawn\)}
\(\mspawn f(x)\) a la même sémantique que \(\mpre f(x)\), mais le calcul de
\(f(x)\) s'exécute sur un nouveau fil d'exécution. Le système de type de Rust garantie que cette
exécution en parallèle est sûre, au sens où elle ne peut pas provoquer de data race.

\(\mspawn f(x)\) est équivalent à \(\mpre f(x)\), et non à \(f(x)\), car ainsi le cycle où l'on a
besoin d'obtenir la valeur de \(f(x)\) n'est pas le même que celui où l'on dispose de la valeur
de \(x\) (et donc où l'on peut lancer le calcul de \(f(x)\)).

\subsection{Sémantique paresseuse}
Lustre est un langage pure; en tant que tel, il ne précise si certaines expressions doivent être
calculées si elles ne sont ensuite pas utilisées, car ça ne changerait rien au résultat. Rustre,
d'un autre côté, permet des appels à des fonctions Rust quelconques, ce qui pourrait avoir des
effets de bord. Pour des opérateurs comme \(e_1 \texttt{->} e_2\), il faut donc choisir si
\(e_1\) est évaluée à chaque itération ou uniquement à la première, c'est-à-dire, est-ce que
l'opérateur \texttt{->} est paresseux. Nous avons choisi de traiter cet opérateur de façon
paresseuse.

\pagebreak
\appendix
\noindent{\huge Annexe}

\section{Grammaire de Rustre}
\begin{grammar}
  <module> ::= (`#![pass(' <int> `)')${}^?$ <node>${}^*_;$

  <node> ::= `#![export]'${}^?$ `node' <ident> <node\_params> `=' `(' <node\_return> `)' <body>

  <node\_params> ::= (<ident> `:' <type>)${}^*_,$

  <node\_return> ::= <ident>${}^*_,$

  <body> ::= <equation>${}^*_,$

  <equation> ::= <ident> `:' <type> `=' <expr>

  <expr> ::= <expr> `when' <ident>
  \alt <expr> `whennot' <ident>
  \alt <expr> `->' <expr>
  \alt <expr> `+' <expr>
  \alt <expr> `-' <expr>
  \alt <expr> `*' <expr>
  \alt <expr> `/' <expr>
  \alt <expr> `\%' <expr>
  \alt `-' <expr>
  \alt `pre' <e5>
  \alt <expr> `as' `float'
  \alt <expr> `>=' <expr>
  \alt <expr> `>' <expr>
  \alt <expr> `<=' <expr>
  \alt <expr> `<' <expr>
  \alt <expr> `==' <expr>
  \alt <expr> `!=' <expr>
  \alt `!' <expr>
  \alt <expr> `&&' <expr>
  \alt <expr> `||' <expr>
  \alt <literal>
  \alt <ident>
  \alt `spawn'$^?$ <ident> `(' (<expr>)${}^*_,$ `)'
  \alt `if' <expr> `{' <expr> `}' `else' `{' <expr> `}'
  \alt `merge' <ident> `{' (<bool> => <expr>)${}^*_,$ `}'
  \alt `(' <expr> `)'

  <type> ::= `int'
  \alt `float'
  \alt `bool'
\end{grammar}

% \section{Règles de typage}
% \begin{figure}[h]
%   \centering
%   \typerule{}{\Gamma,x:\tau\vdash x:\tau}
%   \typerule{\Gamma\vdash e_1:\tint\qquad \Gamma\vdash e_2:\tint\hskip\qquad o\in\{+,*,-,\%,/\}}{\Gamma\vdash e_1 o e_2:\tint}
% \end{figure}
\end{document}
